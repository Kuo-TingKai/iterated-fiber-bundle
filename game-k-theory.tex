\documentclass[12pt]{article}
\usepackage{amsmath, amssymb, amsthm}
\usepackage{geometry}
\geometry{a4paper, margin=1in}
\begin{document}

\title{Using Algebraic \( K \)-Theory in Two-Player Game Strategy Matrices}
\author{}
\date{}
\maketitle

\section*{Introduction}
Two-player game theory and algebraic \( K \)-theory can be connected through
the study of strategy matrices, 
enabling a deeper understanding of the algebraic structures 
behind game equilibria. This document explores how \( K \)-theory can be 
applied to classify, analyze, and simplify strategy matrices in the context
 of two-player games.

\section*{1. Strategy Matrices and Algebraic Structures}
In two-player games, the strategy matrices \( A \) and \( -A^T \) describe 
the payoffs for Players \( A \) and \( B \), respectively. 
Elements \( a_{ij} \) in \( A \) represent Player \( A \)'s payoff 
when choosing strategy \( i \) while Player \( B \) chooses strategy \( j \).
These matrices can be analyzed using algebraic \( K \)-theory by 
treating them as modules, vector bundles, or representations, 
which facilitates the study of their stability, reducibility, 
and invariants.

\section*{2. \( K_0 \)-Theory: Classification and Equivalence of Strategies}
\subsection*{Definition of \( K_0 \)}
\( K_0 \) theory studies the equivalence classes of projective modules. For a ring \( R \), a matrix \( A \) is treated as a module \( M \) over \( R \), and its equivalence class in \( K_0(R) \) indicates its reducibility and rank.

\subsection*{Applications to Game Matrices}
1. A strategy matrix \( A \) can be represented as a projective module 
\( M \) over \( R \), with its rank representing the dimension of
 the strategy space.\\
2. If two matrices \( A \) and \( B \) belong to the same equivalence class
 in \( K_0 \), their strategic properties (e.g., Nash equilibria) may
  exhibit similar characteristics.

\section*{3. \( K_1 \)-Theory: Stability and Determinants of Matrices}
\subsection*{Definition of \( K_1 \)}
\( K_1(R) = GL_\infty(R)/E_\infty(R) \), where \( GL_n(R) \) is 
the general linear group and \( E_n(R) \) is the group of elementary matrices.

\subsection*{Applications to Game Matrices}
1. \textbf{Mixed Strategy Stability:} Solving a mixed-strategy 
Nash equilibrium often involves analyzing the determinant of 
the strategy matrix \( A \). \( K_1 \)-theory helps understand
 the equivalence of matrices under row and column operations.\\
2. \textbf{Invertibility:} A strategy matrix \( A \) 
with \( \det(A) \neq 0 \) belongs to the group of units in \( K_1 \), 
ensuring stable and well-defined Nash equilibria.

\section*{4. \( K_2 \)-Theory: Higher Invariants and Mixed Strategies}
\subsection*{Definition of \( K_2 \)}
\( K_2 \) studies higher-order relations among matrices, 
particularly commutators and Steinberg groups \( St(R) \).

\subsection*{Applications to Game Matrices}
1. \textbf{Mixed Strategy Invariants:} \( K_2 \)-theory provides tools to
 study the combination and interaction of mixed strategies by
  analyzing commutator relations among matrices.\\
2. \textbf{Stable Module Classifications:} If \( A \) is a strategy matrix,
 \( K_2 \) helps classify its stable algebraic properties,
  such as reducibility into block-diagonal forms.

\section*{5. Example: Zero-Sum Game Analysis}
Consider a zero-sum game with the payoff matrix:
\[
A = \begin{bmatrix}
2 & -1 \\
-1 & 1
\end{bmatrix}.
\]

1. \textbf{\( K_0 \) Analysis:}
The module \( M \) corresponding to \( A \) has rank 2.
In \( K_0 \), the module classifies as a stable module without
    reducible components.

2. \textbf{\( K_1 \) Analysis:}
   - The determinant \( \det(A) = 3 \), indicating that \( A \) is 
   invertible and belongs to the unit group in \( K_1 \).

3. \textbf{\( K_2 \) Analysis:}
Higher invariants from \( K_2 \) describe potential interactions 
among mixed strategies, confirming the matrix's stable classification.

\section*{Conclusion}
Algebraic \( K \)-theory provides powerful tools to analyze two-player
 game strategy matrices. By leveraging \( K_0 \), \( K_1 \), and \( K_2 \),
  we can classify strategy spaces, determine stability, and uncover
offers a rich mathematical framework for advancing the study of 
strategic interactions.

\end{document}
