\documentclass[12pt]{article}
\usepackage{amsmath, amssymb, amsthm}
\usepackage{geometry}
\geometry{a4paper, margin=1in}

\begin{document}

\title{Constructing an Iterated Fibre Bundle: A Simple Example}
\author{}
\date{}
\maketitle

\section*{Introduction}
This document demonstrates the explicit construction of an iterated fibre bundle through a simple example. The goal is to construct a two-layer fibre bundle where the base space of the first fibre bundle serves as the total space of the second.

\section*{Construction of the Iterated Fibre Bundle}

We construct the iterated fibre bundle in the following steps:

\subsection*{Step 1: Define the First Fibre Bundle}
Choose the base space $B = S^1$ (a circle).
Let the fibre $F_1 = S^1$.
Define the total space $E_1 = S^1 \times S^1$, which is a torus $T^2$.
Define the projection map $\pi_1: T^2 \to S^1$ as:
  \[
  \pi_1(\theta_1, \theta_2) = \theta_1,
  \]
  where $(\theta_1, \theta_2)$ are points on $T^2$.
Check the fibre structure:
  \begin{itemize}
    \item For any $\theta_1 \in S^1$, the fibre is:
      \[
      \pi_1^{-1}(\theta_1) = \{\theta_1\} \times S^1,
      \]
      which is homeomorphic to $S^1$. Thus, $F_1 \to E_1 \to B$ forms a fibre bundle.
  \end{itemize}

\subsection*{Step 2: Define the Second Fibre Bundle}
Let the total space of the first fibre bundle $E_1 = T^2$ serve as the base space for the second fibre bundle.
Choose the fibre $F_2 = \mathbb{R}$.
Define the total space $E_2 = T^2 \times \mathbb{R}$.
Define the projection map $\pi_2: T^2 \times \mathbb{R} \to T^2$ as:
  \[
  \pi_2((\theta_1, \theta_2), r) = (\theta_1, \theta_2),
  \]
  where $((\theta_1, \theta_2), r)$ are points in $E_2$.
Check the fibre structure:
  \begin{itemize}
    \item For any $(\theta_1, \theta_2) \in T^2$, the fibre is:
      \[
      \pi_2^{-1}((\theta_1, \theta_2)) = \{(\theta_1, \theta_2)\} \times \mathbb{R},
      \]
      which is homeomorphic to $\mathbb{R}$. Thus, $F_2 \to E_2 \to E_1$ forms a fibre bundle.
  \end{itemize}

\subsection*{Step 3: The Iterated Fibre Bundle}
Combining the two layers of fibre bundles, we obtain the iterated fibre bundle:
\[
F_2 \to E_2 \to B,
\]
where $E_2 = T^2 \times \mathbb{R}$ is the total space, $B = S^1$ is the base space, and the final fibre is $F_2 = \mathbb{R}$.

\section*{Verification}
1. The first fibre bundle $F_1 \to E_1 \to B$ is valid because for each $\theta_1 \in S^1$, the fibre $\pi_1^{-1}(\theta_1) \cong S^1$ is a connected space.
2. The second fibre bundle $F_2 \to E_2 \to E_1$ is valid because for each $(\theta_1, \theta_2) \in T^2$, the fibre $\pi_2^{-1}((\theta_1, \theta_2)) \cong \mathbb{R}$ is a trivial fibre.
3. Since fibre bundles can be composed, the resulting iterated fibre bundle $F_2 \to E_2 \to B$ satisfies the definition.

\section*{Conclusion}
This construction explicitly demonstrates a simple example of an iterated fibre bundle, where $F_2 = \mathbb{R}$, $E_2 = T^2 \times \mathbb{R}$, and $B = S^1$.

\end{document}
